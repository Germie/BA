\section{Grundlagen und Stand der Technik}\label{sec:GrundlagenUSdT}

\subsection{Lithium-Ionen-Batterietechnologie}\label{subsec:LIB}

In diesem Kapitel wird ein Überblick über die Lithium-Ionen-Batterietechnologie verschafft. Außerdem wird das thermische und elektrische Verhalten der Zellen erläutert. Zuletzt werden noch Alterungsmechanismen und Temperaturabhängigkeiten betrachtet.

\subsubsection*{Aufbau und Funktionsweise}\label{subsec*:LIBAufbau}

Der Begriff Lithium-Ionen-Batterie umfasst viele verschiedene Batterietechnologien, welche alle auf dem gleichen Wirkprinzip beruhen. Analog zu allen anderen Batterietypen besteht eine Lithium-Ionen-Batterie aus dem Elektrolyten, einem Separator und zwei Elektroden.\\
Nach Konvention wird nach den elektrischen Zuständen beim Entladevorgang die negativ geladene Elektrode als Anode und die positiv geladene Elektrode als Kathode bezeichnet.\\

\begin{figure}[H]
	\begin{overpic}[width=12cm]{figs/LIB_Aufbau_Schematic.eps}
			\put(343,180){\mbox{Sauerstoff}}
			\put(343,155){\mbox{Metall}}
			\put(343,130){\mbox{Lithium-Ion}}
			\put(343,105){\mbox{Kohlenstoff(Graphit)}}
			\put(343,80){\mbox{Separator}}
			\put(343,55){\mbox{Entladen}}
			\put(343,30){\mbox{Laden}}
			\put(240,2){\mbox{Anode}}
			\put(25,2){\mbox{Kathode}}
			\put(135,2){\mbox{Elektrolyt}}
			
	\end{overpic}
	
		\caption[Blah]{Aufbau Lithium-Ionen-Batteriezelle in Anlehnung an Ecker u. Sauer 2013}
	
		\label{fig:LithiumIonAufbau}
\end{figure}
%
%
Wie in Abbildung \ref{fig:LithiumIonAufbau} dargestellt, können die Lithium-Ionen durch das Elektrolyt von der Kathode zur Anode oder umgekehrt ?wandern?. \\
Um die Oberfläche für das Einlagern der Ionen möglichst groß zu gestalten, sind die Materialien der beiden Elektroden hochporös. Dies ermöglicht zudem eine hohe Reaktionsrate. \\
Die Kathode einer \textbf{LIB} (Lithium-Ionen-Batterie) besteht meist aus einem Metalloxid, die Anode aus einer Kohlenstoffmodifikation, oftmals Graphit. Für die Bindung der Elektrodenmaterialien wird häufig Polyvinylidenfluorid (PVFD) in verschiedenen Formen verwendet.\\
Diese Materialien werden dann auf einer dünnen Metallfolie aufgetragen. An der Kathode kommt hierfür Aluminium zum Einsatz, an der Anode wird Kupfer verwendet. Diese Metallfolien dienen gleichzeitig als Stromableiter.\\
Der Separator besteht normalerweise aus einem porösen Polymer. Die Bauteile der Batterie, Elektroden und Separator sind in einem Elektrolyt getränkt. Dieses besteht aus Lithiumsalz das in einem organischem Solvat gelöst ist. Das Solvat wird so gewählt, dass es bei den im Betrieb auftretenden Spannungszuständen trotzdem weitgehend stabil ist.\\
Die Bauform der Batteriezellen ist meist einer von drei etablierten Typen, wie in Tabelle \ref{tab:BauformenZelle} dargestellt ist. Bei allen Typen bestehen die Zellen aus mehreren Lagen von Elektroden-Separator-Elektroden-Stapeln, die je nach Typ geschichtet oder gewickelt werden.\\
Das Gehäuse oder das Verpackungsmaterial der Batteriezellen ist ausschließlich aus Metall. Diese Maßnahme dient der Abdichtung der Zelle, da Wassereintritt die Hydrolyse des Leitsalzes $LiPF_{6}$ zu Fluorwasserstoff anstoßen kann. Zudem verhindert das Metall das Diffundieren des Elektrolyten aus der Zelle nach außen\footcite[Vgl.][S.107-117]{Wohrle2013}.\\

\begin{table}[H]
	\caption{Bauformen der Batteriezellen}
	\label{tab:BauformenZelle}
	\vspace{0.2cm}	
	\begin{tabularx}{\textwidth}{ |X|X|X|  }
		\toprule[1.5pt]
		\textbf{Bauform} & \textbf{Geschichtet} & \textbf{Gewickelt} \\
		\hline\hline
		Zylindrisch &  & Ja\\
		\hline
		Prismatisch & Ja & Ja \\
		\hline
		Pouch-Bag & Ja &  \\
		\bottomrule[1.5pt]
	\end{tabularx}		
\end{table}

Je nachdem ob eine höhere Energiedichte oder Leistungsdichte bei den Batteriezellen gewünscht ist, können die Aktivmaterialschichten den Anforderungen entsprechend ausgelegt werden. Bei dünneren Schichten ist die Leistungsdichte, bei gleichzeitig geringerer Energiedichte, höher\footcite[Vlg.\label{Ecker2013}][S.66-67]{Ecker2013}.\\

\subsubsection*{Batterietechnologien}\label{subsub:BatTechnology}

Der größte Unterschied zwischen den LIB-Technologien liegt in der Materialzusammensetzung der Elektroden.\\
Beginnend mit der Lithium-Kobalt-Oxid-Batterie (LCO) von Sony in den 1990er-Jahren, wurden seitdem auch Lithium-Nickel-Oxid- (LNO) oder Lithium-Mangan-Oxid-Batteriezellen (LMO) entwickelt. LNO und LCO weisen beide eine hohe Kapazität auf, LNO besitzt jedoch eine geringe thermische Stabilität und LCO ist aufgrund des Kobaltgehalts in Sachen Kosten, Sicherheit und Umweltverträglichkeit nicht ideal für den Massenmarkt. LMO ist zwar stabil, löst sich jedoch bei Raumtemperatur teilweise im Elektrolyten. \\
Um diese Nachteile auszugleichen wurden beispielsweise die Lithium-Nickel-Mangan-Kobalt-Batteriezelle (NCM) oder die Lithium-Nickel-Kobalt-Aluminium-Batteriezelle (NCA) entwickelt. Eine weitere Variation ist die Batteriezelle mit einer Kathode auf Eisenphosphatbasis (LFP). Diese besitzt verglichen mit den anderen Varianten eine geringere Spannungslage und gravimetrische Energiedichte. Da aber kein Kobalt verbaut wird, besitzt diese Technologie relevante Umweltvorteile.\\
Auch eine alternative Anoden-Technologie, die sogenannte LTO-Zelle (Lithiumtitanat), kommt aktuell auf den Markt. Sie besitzt zwar eine geringere Energiedichte im Vergleich zu anderen Technologien, weist sich jedoch durch eine hohe Leistungsdichte und Lebensdauer aus\footnote{Vgl. Fußnote \ref{Ecker2013}, Ecker und Sauer 2013}.  \\
In der Industrie setzt sich jedoch die NCM-Kathoden-Technologie durch. Zwar sind aktuell in der Transportbranche alle Technologien vertreten, jedoch stellen Produzenten wie Panasonic ihre NCA-Technologie auf die NCM-basierten Zellen um. In der Gigafactory die Panasonic in Zusammenarbeit mit Tesla in Nevada (USA) seit 2017 in Betrieb hat sollen ab 2025 nur noch NCM-Zellen produziert werden. Die Anoden-Technologie wendet weitgehend Graphit basierte Lösungen an.\\
Diese Entwicklung ist auf die Batterie-Roadmaps der OEM-Hersteller zurückzuführen. In diesen Roadmaps stellen Hochenergie-LIB's mit NCM-Basierten Kathoden die aussichtsreichste Wahl was Kosten und Energiedichte angeht dar\footnote{Vgl. Fußnote \ref{cite:Hettesheimer}, Hettenheimer et al. 2017}. \\
Daher wird in dieser Arbeit die NMC-Kathoden-Technologie betrachtet.%ist das wirklich nötig hier??

\subsubsection*{Elektrische Funktionsweise und Kenngrößen der Lithium-Ionen-Batterietechnologie}\label{subsub:ElekChemFunktion}

Während des Entladevorgangs, was dem Auslagern von Lithium aus der negativen Elektrode entspricht, werden Elektronen ausgegeben. Die Lithium-Ionen wandern durch das Elektrolyt und den Separator zur positiven Elektrode und lagern sich dort ein. Gleichzeitig fließen die Elektronen durch die externe Verbindung über einen Verbraucher zur positiven Elektrode wo Aluminium als Stromableiter dient. Beim Laden wird dieser Prozess umgekehrt\footcite[Vgl.\label{cite:Leuthner}][S. 13-19]{Leuthner.2013}.\\
Der Interkalationsprozess der Lithium-Ionen ist nahezu reversibel. Daher tritt unter normalem Gebrauch meist kein Lithium-Plating auf\footcite[Vgl.][S. 265-270]{DAHN1994}.\\
Beim Lithium-Plating setzt sich Lithium an der Anode ab und verringert so die Kapazität der Batterie.\\


Die gebräuchlichen Kenngrößen der Batteriezelle sind meist die nominale Kapazität, elektrische Energie und Leistung. \\
Die Kapazität ist die Menge an elektrischer Ladung, welche von einer Leistungsquelle während der Entladung geliefert werden kann. Sie wird primär von dem Entladestrom, der Temperatur, der Entladeschlussspannung und der Art und Menge der Aktivmaterialien auf den Elektroden beeinflusst. Die Einheit ist [\textbf{Ah}].\\
Die Energie einer Batteriezelle berechent sich aus der Kapazität und der mittleren Entladespannung und wird in der Einheit [\textbf{Wh}] angegeben. Die Energiedichte bezieht sich auf das Volumen der Batterie und wird in [\textbf{Wh/l}] gerechnet. Die spezifische Energiedichte referenziert die Masse der Batterie und wird dementsprechend in [\textbf{Wh/kg}] angegeben.\\
Um die Leistung zu berechnen wird der Strom mit der Spannung multipliziert, was die Einheit Watt [\textbf{W}] ergibt. Durch den nahezu reversiblen Prozess ist der Wirkungsgrad von Lithium-Ionen-Batterien sehr hoch. Dieser ist nach Gleichung \ref{gl:WirkungsgradLIB} definiert als die Energie die bei der Entladung frei wird geteilt durch die Energie, die beim Laden aufgewendet wird\footnote{Vgl. Fußnote \ref{cite:Leuthner}, Leuthner 2013}.\\

\begin{equation}
	\mbox{Wirkungsgrad} = \frac{Entladeenerige}{Ladeenergie}
	\label{gl:WirkungsgradLIB}
\end{equation}


\subsubsection*{Bauformen der Lithium-Ionen-Batterien}\label{subsub:BauformenLIB}

Wie in Tabelle \ref{tab:BauformenZelle} dargestellt, gibt es drei Bauformen von Lithium-Ionen-Batterien. Da in dieser Arbeit nur die zylindrische und prismatische Bauform relevant ist, werden nur sie hier behandelt.\\
Die zylindrische Zelle besitzt gewickelte Elektroden-Separator-Elektroden-Paare. Analog zu einer klassischen AA-Batterie sind die positiven und negativen Anschlüsse auf jeweils einer der beiden Stirnseiten angebracht. In der konventionellen Batteriezelle wird der Strom durch ein sogenanntes $"Tab"$ aus den Kathoden-Anoden-Paaren entnommen. Der $Tab$ der Kathode oder Anode ist demnach an einer Stelle der Elektrode angebracht. Dadurch muss der Strom zum Teil große Distanzen zurücklegen um über die $Tab$-Verbindung zum elektrischen Leitstück am Ende der Batteriehülle, oder $Can$, zu gelangen. In Kapitel \ref{} wird dieser Prozess näher erläutert.\\ %%%%%%%%%%%%%%%%%%%%%%%% HIER NOCH REFERENCE EINFUEGEN!!!!!!!%%%%
Die Bauform der zylindrischen Zelle wird häufig durch ein Zahlenkürzel angegeben, bei dem die ersten beiden Ziffern den Durchmesser in [mm] vorgeben. Die nächsten beiden Ziffern stehen für die Zellhöhe, wieder in [mm]. Die letzte Ziffer ist eine 0 und schließt die Zahl ab. Zum Beispiel ist eine häufige Bauform die 18650er Zelle mit einem Durchmesser von 18 mm und einer Höhe von 65 mm. Auch viel verwendet werden die 26650er- und die 21700er-Zellgrößen\footcite[Vgl.][]{LionKnowledge2021Zylind}.\\


\begin{figure}[!h]
	\begin{center}
		\begin{overpic}[width=12cm]{figs/Prismatische_und_Zylindrische_Zelle.eps}
		\put(20,250){\mbox{Zylindrische Zelle}}
		\put(225,250){\mbox{Prismatische Zelle}}
		
		\end{overpic}
	\end{center}
	
	
	\caption[Blah]{Prismatische und zylindrische Batteriezelle in Anlehnung an Ecker u. Sauer 2013}
	
	\label{fig:PrismaZylindZelle}
\end{figure}

Der Aufbau der prismatischen Zelle (siehe Abbildung \ref{fig:PrismaZylindZelle}) ist relativ einfach. \\
In diesem Bauformat werden gewickelte oder geschichtete Elektrodenpaare verwendet. Die Spannungs-, bzw. Stromentnahme erfolgt analog zu der konventionellen zylindrischen Batterie-\newline zelle über $"Tabs"$ und wird über an der Kopfseite der Batteriezelle angebrachte Anschlüsse entnommen.\\
Die Größe der prismatischen Zellen variiert stark und wird je nach Anwendungsfall bestimmt\footcite[Vgl.][]{LionKnowledge2021Prisma}. 


\subsubsection*{Alterungsmechanismen}\label{subsub:alterung}

Die Temperaturgradienten die sich beim Benutzen der Batterie in der Zelle ausbilden, können auch Alterungsgradienten hervorrufen. Hinzu kommt, dass starke Temperaturgradienten, die z.B. während hoher C-Raten auftreten, Verformungen der Elektrodenwickel induzieren können\footcite[Vgl.][S.921-927]{Waldmann2015}.\\
Es ist also zu Schlussfolgern, dass eine homogene Temperaturverteilung innerhalb der Zelle von Vorteil ist.
Im weiteren werden drei häufig auftretende Alterungsmechanismen erläutert.\\
Beim Lithium-Plating setzen sich Lithium-Ionen auf der Trennschicht, oder ``Solid Electrolyte Interface" (SEI), zwischen Elektrode und Elektrolyt durch irreversible chemische Reaktionen ab. Diese Ablagerungen verdicken die Trennschicht, was zu einem Anstieg des Stofftransportwiderstands und dadurch zu einem Anstieg des ohm'schen Wiederstands führt. Da die Konzentration der Lithium-Ionen im Elektrolyt auch abnimmt, kommt es außerdem zu einem Kapazitätsverlust.\\
Alterung kann außerdem durch mechanische Spannungen ausgelöst werden. Diese entstehen, wenn Lithium-Ionen sich in die Aktivmaterialien einlagern. Die Spannung innerhalb der Partikel kann hierbei zur Rissbildung führen. Durch die Risse sind die Teile des Aktivmaterials nicht mehr elektrisch angebunden und werden als ``Dead-Lithium`` bezeichnet.\\
Der letzte hier behandelte Alterungsvorgang entsteht aus Dehnvorgängen bei der Lithium-Einlagerung. Diese Belastung kann den Leitruß, einen speziellen Kohlenstoffleiter der Leitpfade zwischen Stromableitern und Partikeln bereitstellt, auftrennen. Dadurch sind die Aktivmaterialpartikel nicht mehr mit dem Stromableiter verbunden. Dieser Alterungsvorgang kann sowohl an Kathode, als auch an der Anode auftreten.\footnote{Vgl. Fußnote \ref{cite:Leuthner}, Leuthner 2013}\\

\subsubsection*{Wärmeentwicklung und Temperatureinfluss auf die Batteriezelle}\label{subsub:waermeundtemperatur}

Die Leistung von LIB's hängt stark von der Zelltemperatur ab. \\
Mit sinkender Temperatur steigt der innere Widerstand der Zelle und die verfügbare Kapazität nimmt ab. Dies führt zu verminderter abnehmbarer Energie und geringerer maximaler Leistung. Bei hoher Zelltemperatur kann jedoch die Sicherheit der Batteriezelle nicht mehr gewährleistet werden und es finden Alterungsprozesse statt. Die Zelltemperatur ist hierbei von der Außentemperatur und der Wärmeentstehung beim Laden bzw. Entladen der Batterie abhängig.\footcite[Vgl.\label{cite:Liu}][S.1001-1010]{Liu2014} \\
Liu et al. stellen Gleichung \ref{gl:cycleheatgeneration} für die Berechnung der Wärmegeneration in der Batteriezelle während der Ladezyklen auf.


\begin{equation}\label{gl:cycleheatgeneration}
		\dot{Q} = I \cdot (U_{OCV} - U_{t}) 
		- I \cdot T \frac{\partial U_{OVC}}{\partial T} 
		- \sum_{i}^{ } \Delta S_{i}^{avg}r_{i}
		- \int \sum_{i}^{ } (\overline{S_{j}} - \overline{S_{j}^{avg}})	\cdot \frac{\partial c_{j}}{\partial t} dv
\end{equation}

\begin{table}[h!]
	\caption{Variablen aus Gleichung \ref{gl:cycleheatgeneration}}
	\label{tab:variablenheatgeneration}
	\vspace{0.2cm}	
	\begin{tabularx}{\textwidth}{ |X|X|  }
		\toprule[1.5pt]
		\textbf{Nummer} & \textbf{Beschreibung} \\
		\hline\hline
		$\dot{Q}$ & Wärmeentstehungsrate \\
		\hline
		$I$ & Batteriestrom\\
		\hline
		$U_{t}$ & Klemmenspannung der Batterie\\
		\hline
		$U_{OCV}$ & Leerlaufspannung der Batterie\\
		\hline
		$T$ & Batterietemperatur\\
		\hline
		$\Delta S_{i}$ & Entropieänderung der $i$-ten Reaktion \\
		\hline
		$r_{i}$ & Reaktionsrate der $i$-ten Reaktion\\
		\hline
		$\overline{S_{j}}$ & Molare Entropie des $j$-ten Teils der Batterie\\
		\hline
		$c_{j}$ & Ionenkonzentration im $j$-ten Teil\\
		\hline
		$v$ & Volumen\\
		\hline
		$X^{avg}$ & Entspricht der Durchschnittskonzentration in einem Teilvolumen\\
		\bottomrule[1.5pt]
	\end{tabularx}		
\end{table}

Der erste Term von Gleichung \ref{gl:cycleheatgeneration} auf der rechte Seite entspricht der ohm'schen Wärmeentstehung, kurz $\dot{Q}_{jou}$. Der zweite Term ist die reversible Entropiewärme oder Reaktionswärme, $\dot{Q}_{re}$. Der dritte Term ist die Wärme aus Nebenreaktionen, $\dot{Q}_{sr}$. Dieser beschreibt die Alterung der Batteriezelle und kann für wenige Testzyklen vernachlässigt werden\footcite[Vgl.][]{Forgez2010}. Der letzte Term, $\dot{Q}_{mix}$, entspricht der Wärme im Mischprozess der Batterie. Die Mischprozesswärme entsteht durch die Bildung und Entspannung von Zellkonzentrationsgradienten. Für dynamische Belastungsprofile ist sie daher signifikant und kann nicht vernachlässigt werden\footcite[Vgl.][]{Thomas2003}. \\
Zusammengefasst verkürzt sich Gleichung \ref{gl:cycleheatgeneration} demnach zu Gleichung \ref{gl:cycleheatshortened}, bzw. Gleichung \ref{gl:cycleheatshort}.

\begin{equation}\label{gl:cycleheatshortened}
	\dot{Q} = I \cdot (U_{OCV} - U_{t})
	- I \cdot T \frac{\partial U_{OVC}}{\partial T}
	- \int \sum_{i}^{ } (\overline{H_{j}} - \overline{H_{j}^{avg}})	\cdot \frac{\partial c_{j}}{\partial t} dv
\end{equation}

\begin{equation}\label{gl:cycleheatshort}
	\dot{Q} = \dot{Q}_{jou} + \dot{Q}_{sr} + \dot{Q}_{mix}
\end{equation}

Nach Gleichung \ref{gl:cycleheatgeneration} ist die ohm'sche Wärme von dem Stromfluss und der Überpotential abhängig. Das Überpotential entsteht durch den Potentialverlust durch einen erhöhten inneren Widerstand der Batteriezelle.\\
Folgend kann der innere Zellwiderstand $R_{in}$ nach Gleichung \ref{gl:innererWiderstand} definiert werden.\footnote{Vgl. Fußnote \ref{cite:Liu}, Lui et al. 2014}

\begin{equation}\label{gl:innererWiderstand}
	R_{in} = \frac{U_{OCV} - U_{t}}{I}
\end{equation}

Der innere Widerstand einer Lithium-Ionen-Batterie wird durch das State-Of-Charge (SOC), die Zellalterung und die Zelltemperatur beeinflusst. Im Allgemeinen sind die Kausalitäten bereits bekannt. Der Widerstand nimmt mit fallender Zelltemperatur zu. Außerdem variiert er mit dem SOC und steigt über die Lebensdauer mit dem Auftreten von Alterungsmechanismen.\footcite[Vgl.][]{Andre.2011}$\;$\footcite[Vgl.][]{Ecker.2012}\\
Interessant ist auch, dass der Ladewiderstand kleiner als der Entladewiderstand ist. Daher wird während dem Ladeprozess weniger ohm'sche Wärme produziert als beim Entladeprozess. Andererseits beeinflusst die Zelldegradation den Ladewiderstand mehr als den Entladewiderstand, wodurch Alterungseffekte beim Berechnen der Wärmeentstehung während Ladezyklen signifikant sind.\\
Bei der Reaktionswärme ist der dominierende Einfluss nach Liu et al. das SOC. In manchen SOC-Bereichen sollten Alterungseffekte berücksichtigt werden und der Temperatureinfluss ist nicht vernachlässigbar.\footnote{Vgl. Fußnote \ref{cite:Liu}, Liu et al. 2014}

\newpage
\subsection{Das Tesla-Patent}

Im Folgenden wird das Patent von Tesla für eine innovative thermische und elektrische Anbindung des Elektrodenwickels an das Batteriegehäuse für die Leistungsabnahme behandelt.\\
Das Patent wurde von Tesla Inc. mit den Erfindern Tsuruta et al am 4. November 2019 eingereicht und am 7. Mai 2020 unter der $Publishing\,No.$ \textbf{US 2020/0144676 A1} veröffentlicht\footcite[Vgl.\label{cite:TeslaPatent}][]{TsurutaTesla2020}.\\
Der Abstract der Veröffentlichung fasst die Erfindung, bei der es sich um eine zylindrische Zelle handelt, wie folgt zusammen: \\
``Eine Zelle eines Energiespeichergeräts mit mindestens einer Elektrode die ohne $Tab$ konstruiert wird und Herstellungsmethoden dieser."\\

In der konventionellen Batteriezelle, ob zylindrisch oder prismatisch, werden die Kathode und Anode mit den positiven und negativen Anschlüssen mithilfe von $Tabs$ verbunden (Beispiel $Tab$ siehe Abbildung \ref{fig:JellyRoll} mit Tabelle \ref{tab:BeschriftungJellyRoll}). \\
Diese $Tabs$ dienen als sogenannte ``Current collectors", d.h. durch sie fließt der gesamte Strom den die Batteriezelle abgibt oder aufnimmt. \\
Da der Widerstand nach Gleichung \ref{gl:WiderstandElektrisch} von der Dichte des Materials ($\rho$), der Länge die der Strom zurücklegt (l) und der Fläche (A) durch den er fließt abhängt, haben Tsuruta et al in ihrem Patent die zurückgelegte Distanz reduziert und die Fläche vergrößert um den Widerstand zu verringern\footnote{Vgl. Fußnote \ref{cite:TeslaPatent}, Tsuruta et al. 2020}.
\begin{equation}
	R = \rho \cdot \frac{l}{A}
	\label{gl:WiderstandElektrisch}
\end{equation}

Im konventionellen Design sind die $Tabs$ entweder in der Mitte oder an einem Ende der Elektrode angebracht. Daher muss der Strom zuerst mindestens die Hälfte, beim $Tab$ am Ende der Elektrode die gesamte Länge der Elektrode zurücklegen. \\
Bei dem innovativen Entwurf ist die maximale Distanz die der Strom zurücklegen muss die Höhe der Elektrode. Abhängend vom Batteriezellenformfaktor entspricht die Höhe der Elektrode typischerweise $5\percent-20\percent$ der Länge. Es folgt, dass der ohm'sche Widerstand des elektrochemischen Zyklus' des innovativen Konzepts 5- bis 20-Mal kleiner ist als der Widerstand des konventionellen Konzepts\footnote{Vgl. Fußnote \ref{cite:TeslaPatent}, Tsuruta et al. 2020}.\\
Ein weiterer Effekt der durch das innovative Konzept hervorgerufen wird ist, dass signifikant weniger Stromabweichungen (Checken mit Jonas) (Stromverteilung auf der Elektrode) auftreten. Stromabweichung ist das Phänomen bei dem manche Elektrodenregionen mehr oder weniger Strom als andere Regionen auf der selben Elektrode während der Zyklendauer leiten.\\
Nach Ohm fließt Strom bevorzugt entlang der Strecke mit dem geringsten Widerstand (siehe Gleichung \ref{gl:OhmsLaw}). In dem konventionellen Design entspricht dies dem Bereich auf der Elektrode nahe des $Tabs$. Die auftretenden Stromabweichungen sind unerwünscht, da es zur Entstehung von lokalen ``Elektroden-Hotspots`` führt. Hier treten Spannungsüberhöhungen auf, welche chemische Reaktionen hervorrufen die die Lebensdauer der Zelle reduzieren. Ein Beispiel einer solchen Reaktion ist Lithium-Plating\footnote{Vgl. Fußnote \ref{cite:TeslaPatent}, Tsuruta et al. 2020}.
\begin{equation}
	I = \frac{V}{R}
	\label{gl:OhmsLaw}
\end{equation}

Die innovativen Konzepte bieten laut Patentschrift auch verbesserte Wärmeentstehungs- und Wärmeübertragungseigenschaften.\\
Nach Gleichung \ref{gl:HeatResistance} ist die ohm'sche Erhitzung in [\textbf{W}] abhängig von dem Strom und dem Widerstand. 
\begin{equation}
	 P \approx I^{2} \cdot R
	 \label{gl:HeatResistance}
\end{equation}

Da der Widerstand \textbf{R} zwischen 5 und 20 mal kleiner ist als beim konventionelle Entwurf ist zu erwarten, dass beim innovativen Design die ohm'sche Wärmeproduktion signifikant reduziert ist.\\
Die trotz der Maßnahmen entstehende Wärme kann effizienter abgeführt werden.
\begin{equation}
	\dot{Q} = \frac{k\cdot A\cdot (T_{2} - T_{1})}{d} 
	\label{gl:HeatTransferEquation}
\end{equation}

Gleichung \ref{gl:HeatTransferEquation} beschreibt den Wärmetransport durch ein wärmeleitendes Medium. \.{Q} ist der Wärmestrom in [\textbf{$\frac{Joule}{Sekunde}$}]. k ist der Wärmeleitkoeffizient des Materials in [\textbf{$\frac{W}{mK}$}].
Die Fläche A in [\textbf{$m^{2}$}] und die Distanz d in [\textbf{$m$}] sind die geometrischen Dimensionen über die der Transfer stattfindet. Der Term $(T_{2}-T_{1})$ entspricht der Temperaturdifferenz über d.\\
In konventionellen Batteriezellen wird die Kontaktfläche zwischen Elektrode und Batteriegehäuse vom $Tab$ realisiert. Beim innovativen Entwurf entspricht der Kontakt zwischen Elektrode und Gehäuse der gesamten Elektrodenlänge. Der Bereich (\textbf{1}) der Elektrode in Abbildung \ref{fig:JellyRoll} befindet sich im Kontakt mit dem Gehäuse und nimmt effektiv $100\percent$ des Durchmessers des Batterie-\newline zylinders ein. Die signifikant größere Kontaktfläche ermöglicht erhöhte Wärmeleitung und dadurch eine optimierte Temperaturkontrolle der Batteriezelle\footnote{Vgl. Fußnote \ref{cite:TeslaPatent}, Tsuruta et al. 2020}.\\

\subsubsection*{Inhalte des Patents}\label{subsub:PatentContents}

In der Patentschrift wird auch auf mögliche Herstellungsvorgänge für die Elektrodenbeschichtung eingegangen. Da diese Arbeit sich mit den physikalischen Vorgängen in der Batteriezelle auseinandersetzt wird die Produktion an dieser Stelle nicht weiter behandelt. Bei Interesse an der Produktion wird die Patentschrift als weiterführende Literatur empfohlen.\\
Es werden verschiedene Konfigurationen der Elektrodenwickel beschrieben. In Abbildung \ref{fig:JellyRoll} ist ein Beispiel für eine ``Jelly-Roll`` zu sehen. \\
Die Kathode oder Anode (\textbf{3}) wird mit dem Aktivmaterial beschichtet. Dabei kann der leitende Teil an dem unteren Ende der Elektrode (\textbf{1}) durch eine isolierende Schicht (\textbf{2}) von dem Aktivmaterial getrennt werden. Diese Schicht ist in manchen Konfigurationen des Patents jedoch nicht vorhanden.\\
Zwischen die beiden Elektroden wird eine Trennschicht eingebracht (\textbf{4}). Da die Elektroden gewickelt werden, wird nach der Anode oder Kathode (\textbf{5}) eine weitere, äußere Trennschicht (\textbf{6}) eingebracht. Diese 4 Schichten werden dann um eine zentrale Achse \textbf{AA'} gewickelt. (\textbf{7}) zeigt das $Tab$ der Anode oder Kathode, welches mit der oberen Seite des Gehäuses verbunden wird.\\
\begin{figure}[H]
	\begin{center}
		\begin{overpic}[width=14 cm]{figs/JellyRollTeslaPatent.eps}
			\put(0,-2){(\textbf{1})}
			\put(-9,27){(\textbf{2})}
			\put(-11,132){(\textbf{3})}
			\put(-14,220){(\textbf{4})}
			\put(-8,270){(\textbf{5})}
			\put(35,302){(\textbf{6})}
			\put(392,232){(\textbf{7})}
			\put(292,270){\textbf{AA'}}
		\end{overpic}
	\end{center}
	
	
	\caption[Blah]{Aufbau des Elektrodenwickels in der zylindrischen Zelle in Anlehnung an Tesla Patent, Tsuruta et al. 2020}
	
	\label{fig:JellyRoll}
\end{figure}

\begin{table}[h!]
	\caption{Beschriftung von Abbildung \ref{fig:JellyRoll}}
	\label{tab:BeschriftungJellyRoll}
	\vspace{0.2cm}	
	\begin{tabularx}{\textwidth}{ |X|X|  }
		\toprule[1.5pt]
		\textbf{Nummer} & \textbf{Beschreibung} \\
		\hline\hline
		(1) & Leitfähiges Material \\
		\hline
		(2) & Isolierendes Material\\
		\hline
		(3) & Kathode/Anode\\
		\hline
		(4) & Innere Trennschicht\\
		\hline
		(5) & Anode/Kathode\\
		\hline
		(6) & Äußere Trennschicht\\
		\hline
		(7) & $Tab$\\
		\bottomrule[1.5pt]
	\end{tabularx}		
\end{table}

\subsubsection*{Der Kontakt zwischen Elektrodenwickel und Cap}\label{subsub:JellyrollCapContact}

Damit der leitende Teil der Elektrode Kontakt zum entsprechenden Pol des Batteriegehäuses hat, wird ein Ende des Batteriezylinders mit einer $Cap$ versehen. Das $Cap$ besteht aus einem leitenden Material, zum Beispiel einer Nickel-Legierung\footnote{Vgl. Fußnote \ref{cite:TeslaPatent}, Tsuruta et al. 2020, S. 1}.\\
Dieser Deckel kann für eine optimierte Anbindung des Elektrodenwickels in verschiedenen Konfigurationen vorhanden sein. Diese Modifikationen der $Cap$ können unter anderem Nuten, Erhebungen und Aussparungen, sowie andere nicht beschriebene Merkmale sein.\\
Der Kontakt zwischen Elektrodenwickel und $Cap$ kann mit mehreren Konzepten verwirklicht werden. Einer davon ist eine Pressverbindung, bei der das $Cap$ und die Kathode oder Anode durch Druck miteinander in Berührung kommen.\\
Damit der Verbindung zwischen $Cap$ und Kathode, bzw. Anode mehr Oberfläche verfügbar ist, kann das $Cap$ so ausgelegt sein, dass es mit dem leitenden Teil der Elektrode (Abbildung \ref{fig:JellyRoll} (\textbf{1})) einen möglichst schlüssigen Kontakt hat. Um die benötigte Presskraft aufzubringen kann oberhalb des Elektrodenwickels ein isolierendes Material eingebracht werden, welches die Elektrode gegen die $Cap$ presst. Die Verbindung kann auch durch Anschwellen der Anode durch den Elektrolyten oder durch Laser- oder Ultraschallschweißen realisiert werden.\\
Für einen möglichst guten Kontakt können wie in Abbildung \ref{fig:CapGrooved} Nuten in das $Cap$ gefräst werden. Die Nuten sollten in der Größenordnung des leitenden Elektrodenteils sein. In der Patentschrift sind hier zwischen 0.01 mm und 0.1 mm angegeben.\\

\begin{figure}[H]
	\begin{center}
		\begin{overpic}[width=14 cm]{figs/CapGroovedTeslaPatent.eps}
			\put(77,210){(\textbf{1})}
			\put(145,225){(\textbf{1})}
			\put(400,103){(\textbf{2})}
		\end{overpic}
	\end{center}
	
	
	\caption[Blah]{Beispielhafter Deckel oder Cap des Batteriezellgehäuses in Anlehnung an Tesla Patent, Tsuruta et al. 2020}
	
	\label{fig:CapGrooved}
\end{figure}

\begin{table}[h!]
	\caption{Beschriftung von Abbildung \ref{fig:CapGrooved}}
	\label{tab:BeschriftungCapGrooved}
	\vspace{0.2cm}	
	\begin{tabularx}{\textwidth}{ |X|X|  }
		\toprule[1.5pt]
		\textbf{Nummer} & \textbf{Beschreibung} \\
		\hline\hline
		(1) & Nut \\
		\hline
		(2) & Kontaktfläche\\
		\bottomrule[1.5pt]
	\end{tabularx}		
\end{table}

Wie in Abbildung \ref{fig:BottomConnection} b) gezeigt kann die Verbindung zwischen Elektrode und Batteriegehäuse auch ohne eine spezielle $Cap$ hergestellt werden. Hierzu werden die leitenden Teile des \newline Elektrodenwickels zum Teil miteinander oder direkt mit dem Gehäuse verbunden. \\

\begin{figure}[H]
	\begin{center}
		\begin{overpic}[width=14 cm]{figs/BottomElectrodeConnector.eps}
			\put(-15,240){\textbf{a)}}
			\put(-15,101){\textbf{b)}}
			\put(190,130){\textbf{(1)}}
			\put(160,260){\textbf{(2)}}
			\put(197,260){\textbf{(2)}}
			\put(8,120){\textbf{(2)}}
			\put(374,120){\textbf{(2)}}
			\put(400,160){\textbf{(3)}}
			\put(169,-5){\textbf{(4)}}
			\put(70,130){\textbf{(4)}}
			\put(300,130){\textbf{(4)}}
		\end{overpic}
	\end{center}
	
	
	\caption[Blah]{Verbindungsmöglichkeiten des Elekrodenwickels mit dem Batteriegehäuse nach Tesla Patent, Tsuruta et al. 2020\newline a) Verbindung mit Cap\newline b) Verbindung ohne Cap}
	%\caption*{a) Verbindung mit Cap, b) Verbindung ohne Cap}
	
	\label{fig:BottomConnection}
\end{figure}

\begin{table}[h!]
	\caption{Beschriftung von Abbildung \ref{fig:BottomConnection}}
	\label{tab:BeschriftungBottomConnection}
	\vspace{0.2cm}	
	\begin{tabularx}{\textwidth}{ |X|X|  }
		\toprule[1.5pt]
		\textbf{Nummer} & \textbf{Beschreibung} \\
		\hline\hline
		(1) & Leitender Elektrodenteil \\
		\hline
		(2) & Elektrodenwickel\\
		\hline
		(3) & Cap\\
		\hline
		(4) & Kontaktfläche\\
		\bottomrule[1.5pt]
	\end{tabularx}	
\end{table}

Imperfektionen können durch Fehler im Herstellungsprozess entstehen. Leitendes Material kann über sich selbst gefaltet sein und dadurch den Kontakt zwischen Elektrodenwickel und $Cap$, bzw. Gehäuse negativ beeinflussen. Um diesen Effekten entgegen zu wirken können  wie in Abbildung \ref{fig:BottomConnectorTypes} gezeigt ist Abschnitte des leitenden Teils der Elektrode entfernt werden. Mögliche Vorgehensweisen sind die Entfernung von Material in festen Abständen (Abbildung \ref{fig:ConductivePortionTypes} a) ) oder in größer werdenden Abständen (Abbildung \ref{fig:ConductivePortionTypes} b) ). Bei b) kann die Anordnung so gewählt werden, dass nach der Wicklung der Elektrode die leitenden Teile ein rotationssymmetrisches Muster bilden.\footnote{Vgl. Fußnote \ref{cite:TeslaPatent}, Tsuruta et al. 2020, S. 5}\\

\begin{figure}[H]
	\begin{center}
		\begin{overpic}[width=14 cm]{figs/ConducticePortionTypes.eps}
			\put(9,146){\textbf{(1)}}
			\put(280,2){\textbf{(1)}}
			\put(113,144){\textbf{(2)}}
			\put(143,-5){\textbf{(2)}}
			\put(186,234){\textbf{(3)}}
			\put(313,105){\textbf{(3)}}
			\put(-15,240){\textbf{a)}}
			\put(-15,101){\textbf{b)}}
		\end{overpic}
	\end{center}
	
	
	\caption[Blah]{Verschiedene Varianten der Auslegung des leitenden Teils der Elektrode nach Tesla Patent, Tsuruta et al. 2020}
	
	\label{fig:ConductivePortionTypes}
\end{figure}

\begin{table}[h!]
	\caption{Beschriftung von Abbildung \ref{fig:ConductivePortionTypes}}
	\label{tab:ConductivePortionTypes}
	\vspace{0.2cm}	
	\begin{tabularx}{\textwidth}{ |X|X|  }
		\toprule[1.5pt]
		\textbf{Nummer} & \textbf{Beschreibung} \\
		\hline\hline
		(1) & Isolierende Schicht \\
		\hline
		(2) & Leitender Teil\\
		\hline
		(3) & Elektrode\\
		\bottomrule[1.5pt]
	\end{tabularx}		
\end{table}

Ist zwischen Elektrode und Batteriegehäuse ein $Cap$ eingebracht kann diese so ausgelegt werden, dass der leitende Teil der Elektrode einfach mit der $Cap$ in Verbindung gebracht wird. \\
Abbildung \ref{fig:BottomConnectorTypes} a), b) und c) zeigen verschiedene Ansätze für die Verbindung mit dem leitenden Teil. \\
Bei a) kann der leitende Teil wie in Abbildung \ref{fig:ConductivePortionTypes} b) dargestellt ist angebracht werden. Das $Cap$ ist mit konzentrischen Nuten versehen, die wie in dem Querschnitt zu erkennen ist, mit einer 90°-Kante versehen ist.\\
C) verfolgt das gleiche Konzept wie a), jedoch ist hier die Auslegung der Nut spiralförmig gewählt. So kann die spiralförmige Wickelung der zu verbindenden Elektrode passgenau an der $Cap$ angebracht werden.

\begin{figure}[H]
	\begin{center}
		\begin{overpic}[width=14 cm]{figs/BottomConnectorTypes.eps}
			\put(-10,240){\textbf{a)}}
			\put(205,240){\textbf{b)}}
			\put(100,100){\textbf{c)}}
		\end{overpic}
	\end{center}
	
	
	\caption[Blah]{Verschiedene Cap-Varianten mit Schnittansicht nach Tesla Patent, Tsuruta et al. 2020}
	
	\label{fig:BottomConnectorTypes}
\end{figure}

\newpage

\subsection{Batteriesimulation\label{sub:SDTsimulation}}

\subsubsection*{Das Simulationsprogramm}

Um die prismatische Batterie zu modellieren und simulieren sind die Programme ANSYS\textsuperscript{\textregistered} und COMSOL Multiphysics\textsuperscript{\textregistered} geeignet.\\
COMSOL Multiphysics\textsuperscript{\textregistered} (auch bekannt als FEMLAB vor 2005) ist ein kommerzielles \newline Simulationsprogramm auf Basis der Finite-Elemente-Methode\footcite[Vgl.\label{cite:comsolwebsite}][]{COMSOLWebsite2021}. Das Programm wird vermehrt in analytischer Elektrochemie eingesetzt\footcite[Vgl.\label{cite:Dickinson}][S.74]{Dickinson.2014}. Da COMSOL\textsuperscript{\textregistered} dem Benutzer erlaubt über ``physics interfaces`` direkt auf die partiellen Differentialgleichungen des Simulationsmodells Einfluss zu nehmen, ist es für Benutzer kontrollierte Physik-Konfigurationen gut geeignet. Nach Klymenko et al. sind die Resultate aus COMSOL\textsuperscript{\textregistered} mit entsprechender Benutzerexpertise von hohem Gütegrad\footcite[Vgl.][]{Klymenko.2013}.\\
ANSYS\textsuperscript{\textregistered} stellt eine gute Alternative für die Simulation dar. Aufgrund seiner effizienten Solver ist es vor allem für industrielle Anwendungen geeignet, während COMSOL\textsuperscript{\textregistered} sich im Forschungsbereich als anwendungsfreundlicher erweist. Beide Programme eignen sich gut für die Simulation von Lithium-Ionen Batterien. Jedoch ermöglicht COMSOL Multiphysics\textsuperscript{\textregistered} im Vergleich zu Ansys\textsuperscript{\textregistered} einen flexibleren Modellaufbau, weshalb es für die Simulation in dieser Arbeit ausgewählt wurde.\footcite[Vgl.][S.196]{Salvi.2010}\\

\subsubsection*{Mathematische Grundlage der Lithium-Ionen Batterie Simulation EVENTUELL RAUSNEHMEN}

Die Simulation von Lithium-Ionen-Batterien beruht auf verschiedenen elektrochemischen Phänomenen: \\ Die Kopplung der Erhaltung der Ladung und des Stromflusses im Elektrolyten und den Elektroden, die Massenerhaltung für die Lösungsspezies im Elektrolyten und die Impulserhaltung zusammen mit der gesamten Masse in einer Lösung oder Mischung. Diese Phänomene werden von partiellen Differentialgleichungen (PDE's) beschrieben. Für diese Gleichungen existiert keine analytische Lösung. Daher müssen sie mittels numerischer Annäherung gelöst werden.\footnote{Vgl. Fußnote \ref{cite:Dickinson}, Dickinson et al. 2014} \\
Die Ladungserhaltung wird nach Gauss mit Gleichung \ref{gl:GaussLaw} beschrieben. Der Massentransport folgt nach Nernst-Planck mit Gleichung \ref{gl:Nernst-Planck}, ist aber auch von der Massenerhaltung nach Gleichung \ref{gl:masscontinuity} abhängig.

\begin{eqnarray}
	- \nabla \cdot (\varepsilon \nabla \varphi) = \rho \label{gl:GaussLaw} \\
	\mathbf{N_{i}} = -D_{i} \nabla c_{i} - z_{i} u_{i} c_{i} \nabla \varphi + c_{i}\mathbf{u} \label{gl:Nernst-Planck} \\
	\frac{\partial c_{i}}{\partial t} + \nabla \cdot \mathbf{N_{i}} = R_{i} \label{gl:masscontinuity}
\end{eqnarray}

Die Gleichung \ref{gl:GaussLaw} bis \ref{gl:masscontinuity} ergeben zusammen die Nernst-Planck-Poisson Gleichungen. Diese beschreiben die Ladungs- und Massentransportseigenschaften einer unendlich verdünnten Elektrolytlösung. Die Ein- und Auslagerung von Lithium-Ionen in die Aktivschicht kann mittels geeigneter Randbedingungen modelliert werden.\footnote{Vgl. Fußnote \ref{cite:Dickinson}, Dickinson et al. 2014 , S. 72}

\begin{table}[H]
	\caption{Variablen aus Gleichungen \ref{gl:GaussLaw} bis \ref{gl:masscontinuity}}
	\label{tab:pdevariables_Gauss_Nernst_Mass}
	\vspace{0.2cm}	
	\begin{tabularx}{\textwidth}{ |X|X|X|  }
		\toprule[1.5pt]
		\textbf{Nummer} & \textbf{Beschreibung} & \textbf{Einheit}\\
		\hline\hline
		$\varepsilon$ & Dielektrizitätskonstante & $F \cdot m^{-1}$ \\
		\hline
		$\varphi$ & Potential & $V$\\
		\hline
		$\rho$ & Ladungsdichte & $C \cdot m^{-3}$\\
		\hline
		$c_{i}$ & Konzentration der Spezies i & $mol \cdot m^{-3}$\\
		\hline
		$\mathbf{N_{i}}$ & Fluss der Spezies i & $mol \cdot m^{-2} \cdot s^{-1}$\\
		\hline
		$D_{i}$ & Diffusionskoeffizient der Spezies i & $m^{2} \cdot s^{-1}$\\
		\hline
		$z_{i}$ & Ladezahl der Spezies i & $[-]$\\
		\hline
		$u_{i}$ & Mobilität der Spezies i & $m^{2} \cdot V^{-1} \cdot s^{-1}$\\
		\hline
		$\mathbf{u}$ & Geschwindigkeit & $m \cdot s^{-1}$\\
		\hline
		$R_{i}$ & Massenquelle der Spezies i & $mol \cdot m^{-3} \cdot s^{-1}$\\ 
		\bottomrule[1.5pt]
	\end{tabularx}		
\end{table}

Die Nernst-Planck-Poisson Gleichungen sind nichtlinear und weisen mehrere Zeit- und Längendimensionen auf. Daher ist die vollständige Lösung dieser Gleichungen unter realistischen Bedingungen in der Praxis nicht gut umzusetzen. Um trotzdem verwertbare Simulationsergebnisse zu erhalten können Vereinfachungen angenommen werden:\\
Eine allgemein anwendbare Vereinfachung ist die Annahme der Elektroneutralität auf Skalen größer als Nanometer (Gleichung \ref{gl:electroneutrality}).\footnote{Vgl. Fußnote \ref{cite:Dickinson}, Dickinson et al. 2014, S. 72}

\begin{equation}\label{gl:electroneutrality}
	\sum_{i} z_{i} c_{i} = 0
\end{equation}

Unter Annahme der Elektroneutralität und kleiner absolute Konzentrationsgradienten des ladungstragenden Elektrolyten folgt der Elektrolytstrom Gleichung \ref{gl:ohmslawcomplex} nach Ohm mit Rücksicht auf eine näherungsweise konstante Leitfähigkeit (Gleichung \ref{gl:constantconductivity}).

\begin{eqnarray}
	-\nabla \cdot (\sigma_{soln} \nabla \varphi) = Q \label{gl:ohmslawcomplex} \\
	\sigma_{soln} \approx F \cdot \sum_{i} z^{2}_i u_{i} c_{i} \label{gl:constantconductivity}
\end{eqnarray}

\begin{table}[H]
	\caption{Variablen aus Gleichungen \ref{gl:ohmslawcomplex} und \ref{gl:constantconductivity}}
	\label{tab:Variablesohmscomplex_constantcond}
	\vspace{0.2cm}	
	\begin{tabularx}{\textwidth}{ |X|X|X|  }
		\toprule[1.5pt]
		\textbf{Nummer} & \textbf{Beschreibung} & \textbf{Einheit}\\
		\hline\hline
		$\sigma$ & Leitfähigkeit & $S \cdot m^{-1}$ \\
		\hline
		$Q$ & Ladequelle & $A \cdot m^{-3}$\\
		\hline
		$F$ & Faraday Konstante & $[Ah \cdot mol^{-1}]$\\
		\hline
		$R$ & Gaskonstante & $J \cdot K^{-1} \cdot mol^{-1}$\\
		\hline
		$T$ & Temperatur & $K$\\
		\bottomrule[1.5pt]
	\end{tabularx}		
\end{table}

Nach Ohm entspricht das Verhältnis von Stromfluss zum elektrischen Feld der Leitfähigkeit. Größere Elektrolytkonzentration im Vergleich zur Reaktantkonzentration erhöht die Leitfähigkeit des Elektrolyten, wodurch das elektrische Feld für beliebige Stromstärken gegen Null tendiert. Daher wird, wenn eine große Menge an unterstützendem Elektrolyt vorhanden ist, die Konzentration dieses als ``effektiv unendlich`` angenommen. Das bedeutet, dass der Massentransport des Reaktants nur als Diffusion, oder an den relevanten Stellen als Konvektion stattfindet. Gleichung \ref{gl:FicksLaw} nach Fick beschreibt hier die Konvektion für ein inkompressible Strömung.\footnote{Vgl. Fußnote  \ref{cite:Dickinson}, Dickinson et al. 2014, S. 72}

\begin{equation}\label{gl:FicksLaw}
	\frac{\partial c_{i}}{\partial t} = D_{i} \nabla^{2} c_{i} - \mathbf{u} \cdot \nabla c_{i} 
\end{equation}

\subsubsection*{COMSOL Multiphysics\textsuperscript{\textregistered}}\label{subsub:comsolMulti}

Der Term ``Multiphysics`` beschreibt die Zusammenführung und Kopplung verschiedener physikalischer Phänomene. In Praxis ist es die Zusammenstellung diverser partieller Differentialgleichungen in einem Modell. \\
Beispielsweise kombiniert die Analyse von Brennstoffzellen Strömungsmechanik, Massentransport, Wärmeübertragung und Ladungsübertragung. In COMSOL\textsuperscript{\textregistered} werden die benötigten Differentialgleichungen in ``physics interfaces`` implementiert. Diese bestehen aus verfügbaren Packeten von physikalischen Gleichungen. Die Randbedingungen können auch in den Interfaces gewählt werden.\\
Klassisch werden in der Modellierung von elektrochemischen Energiespeichern die folgenden Interfaces eingebunden:\footnote{Vgl. Fußnote  \ref{cite:Dickinson}, Dickinson et al. 2014, S. 72}

\begin{itemize}
	\item Primary/Secondary/Tertiary Current Distribution
	\item Electroanalysis
\end{itemize}

In COMSOL\textsuperscript{\textregistered} existiert bereits ein mathematisches Modell einer Lithium-Ionen-Batterie. Dieses ist ausreichend für die Simulation, muss jedoch noch um den thermischen Aspekt erweitert werden.\\

\subsubsection*{Modellierung einer Lithium-Ionen-Batterie in COMSOL\textsuperscript{\textregistered} mit thermischen Effekten}\label{subsub:modellthermalLIB}

Um Lithium-Ionen-Batterien zu simulieren, müssen thermische und elektrochemische Vorgänge simultan berechnet werden. Aufbauend von den Modellen von Shepherd\footcite[Vgl.][]{Shepherd.1965} und Nasar und Unnewehr\footcite[Vgl.][]{Nasar.1982}, welche nur Ladezyklen mit konstantem Stromfluss abbilden können, beschreiben Fink und Kaltenegger ein elektrothermisches Modell (ET) das die dynamische Entspannung und das zeitabhängige Verhalten der Batterie aufgrund von Stromimpulsen beschreiben kann.\footcite[Vgl.\label{cite:fink}][S. 105-124]{Fink.2014} \\
Das elektrochemische Modell der Batterie bauen Fink und Kaltenegger auf dem ein-dimensionalen Modell von Doyle\footcite[Vgl.][S. 1890-1903]{Doyle.1996} auf. Das Modell kann die zeitliche Ausbildung und räumliche Verteilung von elektronischen und ionischen Potentialen, Lithiumkonzentrationen in Feststoffen und dem Elektrolyten, sowie die Temperatur für arbiträre Lastprofile abbilden.\footnote{Vgl. Fußnote \ref{cite:fink}, Fink und Kaltenegger 2014, S. 106}\\
Im Folgenden wird zunächst ein Überblick über die mathematischen Modelle gegeben. Die Simulationsumgebung, Randbedingung, Materialien und Parameter werden in Kapitel [][][]REFERENZ SIMULATIONSKAPITEL %\ref{section:simulation}SIMULATIONKAPITEL 
ausgeführt.\newline

\underline{Stromabnehmer}\\
\\
Das elektronische Potential $\Phi_{ele}$ wird mit Hilfe der Erhaltungsgleichung elektrischer Ladung, Gleichung \ref{gl:conservationOfCharge}, in beiden Modellen (ET und EC) berechnet.

\begin{equation}
	\nabla \cdot \vec{i}_{ele} = 0 \label{gl:conservationOfCharge}
\end{equation}

\begin{equation}
	\vec{i}_{ele} = - \sigma_{c} \nabla \Phi_{ele} \label{gl:Stromflussdichte}
\end{equation}

\begin{equation}
	\frac{\partial (\rho_{c} c_{p,c} T)}{\partial t} + \nabla \cdot \vec{\dot{q}} = -\vec{i}_{ele} \cdot \nabla \Phi_{ele} \label{gl:Energiegleichung}
\end{equation}

Gleichung \ref{gl:Stromflussdichte} beschreibt hier die Stromdichte, Gleichung \ref{gl:Energiegleichung} die Energiebalance.

\begin{equation}
	\vec{\dot{q}} = \ \lambda_{c} T \label{gl:Waermeleitungsstrom}
\end{equation}

Der Wärmeleitstrom wird in Gleichung \ref{gl:Waermeleitungsstrom} aufgestellt.\\

\begin{figure}[H]
	\begin{center}
		\begin{overpic}[width=14 cm]{figs/LayerStructureECETModel.eps}
			\put(160,16){Reaktionszone}
			\put(160,74){Reaktionszone}
			\put(160,133){Reaktionszone}
			%\put(10,120){Aktivschicht}
			\put(10,74){Aktivschicht}
			\put(10,16){Aktivschicht}
			\put(10,103){\textbf{-} Stromabnahme}
			\put(10,45){\textbf{+} Stromabnahme}
			\put(306,103){\textbf{-} Stromabnahme}
			\put(306,45){\textbf{+} Stromabnahme}
			\put(306,90){\textbf{-} Elektrode}
			\put(306,58){\textbf{+} Elektrode}
			\put(306,74){Separator}
			\put(306,31){\textbf{+} Elektrode}
			\put(306,15){Separator}
			\put(306,116){\textbf{-} Elektrode}
		\end{overpic}
	\end{center}
	
	
	\caption[Schichtenmodell der ET und EC Modelle nach Fink 2014]{Vgl. Schichtenmodell der EC- und ET-Modelle nach Fink 2014}
	
	\label{fig:LayermodelFink}
\end{figure}

\underline{Reaktionszone}\\
\\

Die Reaktionszone besteht wie in Abbildung \ref{fig:LayermodelFink} gezeigt aus positiver und negativer Elektrode und einem Separator. \\
Im elektrothermischen Modell (ET) sind die drei Teile zu einer Aktivschicht vereint. Hier befindet sich im Modell eine Wärmequelle. Die Reaktionsstromdichte wird in dem Interface Aktivschicht/Stromabnehmer aus einer empirischen Gleichung berechnet. Die Energiegleichung in der Aktivschicht ist in Gleichung \ref{gl:AktivschichtET} beschrieben.

\begin{equation}
	\frac{\partial (\rho_{al} c_{p,al} T)}{\partial t} + \nabla \cdot \vec{\dot{q}} = \frac{i_{emp} ( V_{oc,emp} - \Delta \Phi_{ele} )}{L_{al}} \label{gl:AktivschichtET}
\end{equation} 

Der Wärmeleitstrom folgt nach Gleichung \ref{gl:Waermeleitungsstrom} mit Gleichung \ref{gl:WaermeleitungsstromExplizit}.

\begin{equation}
	\vec{\dot{q}} = - \left( \begin{array}{ccc} \lambda_{al,x} & 0 & 0\\ 0 & \lambda_{al,y} & 0 \\ 0 & 0 & \lambda_{al,z}\\ \end{array} \right) \nabla T \label{gl:WaermeleitungsstromExplizit}
\end{equation}
 
 Die Variablen $i_{emp}$ und $V_{oc,emp}$ werden aus den Interfaceverbindungsgleichungen nach Fink und Kaltenegger berechnet.\footnote{Vgl. Fußnote \ref{cite:fink}, Fink und Kaltenegger 2014,  S. 110}
 
 
 
 
 

