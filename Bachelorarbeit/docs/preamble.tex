% This file contains your LaTeX preamble. A preamble is a part of your document where all required packages and macros can be defined. This needs to be done before the \begin{document} command.



% Documentclass:
% Standard LaTeX classes are: article, book, report, slides, and letter. These cover the basis, but are not best. More advanced users might want to try out the KOMA classes or the memoir class. Optional arguments: 10pt. The font size of the main content is set to 10pt with the option between [].

\documentclass[a4paper, 11pt, bibtotocnumbered, liststotocnumbered, toc=listofnumbered, captions=nooneline, fleqn]{scrartcl} 
\ReplacePackage{scrpage2}{scrlayer-scrpage}
\usepackage{scrpage2}
%Inhaltsverzeichnis mit im Inhaltsverzeichnis
\setuptoc{toc}{numbered}

%Schriftart
\usepackage{helvet}
\renewcommand{\familydefault}{\sfdefault}

%Deutsches Sprachpaket
\usepackage[ngerman]{babel}
\usepackage[utf8]{inputenc}

%Verbessertes Schriftbild
\usepackage{microtype}

% Geometry:
% The papersize of the document is defined with the geometry package. Here, the size is set to A4 with a4paper. Other possibilities are a5paper, b5paper, letterpaper, legalpaper and executivepaper.
\usepackage[margin=2.5cm]{geometry}
\usepackage[bottom, hang]{footmisc} %Fußnoten nicht einrücken und am Seitenende
\usepackage{longtable} %Tabellen über mehrere Seiten

% AMS math packages:
% Required for proper math display.
\usepackage{amsmath,amsfonts,amsthm}


% Graphicx:
% If you want to include graphics in your document, the graphicx package is required.
\usepackage{graphicx}
\usepackage[abs]{overpic}

% Booktabs:
% The booktabs package is needed for better looking tables. 
\usepackage{booktabs}

% SIunitx:
% The SIunitx package enables the \SI{}{} command. It provides an easy way of working with (SI) units.
\usepackage[decimalsymbol=comma]{siunitx}

% URL:
% Clickable URL's can be made with this package: \url{}.
\usepackage{url}


% Caption:
% For better looking captions. See caption documentation on how to change the format of the captions.
\usepackage{caption}

%Float Package zur Positionierung von Abbildungen
\usepackage{float}
	
%Referenzen Zählweise
\usepackage{chngcntr}
\counterwithin{figure}{section}
\counterwithin{table}{section}
%\counterwithin{equation}{section} Formel durchgehend nummeriert!


%Bild anstelle von Abbildung in der Bildbeschriftung
%\addto\captionsngerman{\renewcommand{\figurename}{Bild}}

%Bildbeschriftung in Fett
\usepackage{caption}
\captionsetup[figure]{labelfont={bf,small}, font={bf,small}}
\captionsetup[table]{labelfont={bf,small}, font={bf,small}}

%Biblatex
%zitieren mit Footcite, Anpassung des Zitierstils gemäß PEM vorgabe
\usepackage[backend=biber, citestyle=authortitle, bibstyle=authortitle, isbn=false, doi=false, url=false, block=space, maxcitenames=2, maxbibnames=99, backref=false, giveninits=true, dashed=false]{biblatex}
\setlength{\bibitemsep}{1em}
\setlength{\bibhang}{2em}
\DefineBibliographyStrings{german}{%
	andothers = {et al.},
}
\DeclareFieldFormat %Klammern um den Kurztitel
[book,article,inbook,incollection,inproceedings,patent,thesis,unpublished,misc,manual]
{citetitle}{\mkbibparens{#1\isdot}}


\renewcommand*{\nametitledelim}{\addspace} %kein Komma hinter dem Autor Namen
\usepackage{xpatch}%Hinzufügen des Erscheinungsjahr in Zitation
\xapptobibmacro{cite}{\setunit{\nametitledelim}\printfield{year}}{}{}

\renewbibmacro*{title}{%
	\ifboolexpr{
		test {\iffieldundef{title}}
		and
		test {\iffieldundef{subtitle}}
	}
	{}
	{\printtext[title]{%
			\printfield[parens]{shorttitle}%
			\setunit{\newline}%
			\printfield[titlecase]{title}%
			\setunit{\subtitlepunct}%
			\printfield[titlecase]{subtitle}}%
		\newunit}%
	\printfield{titleaddon}}

\DeclareFieldFormat %keine Anführungszeichen im Titel
[book,article,inbook,incollection,inproceedings,patent,thesis,unpublished,misc,manual]
{title}{#1\isdot}
\DeclareFieldFormat %keine Klammern beim Jahr
[article,inbook,incollection,inproceedings,patent,thesis,unpublished,misc,manual]
{year}{#1\isdot}

\xpretobibmacro{author}{\mkbibbold\bgroup}{}{}
\xapptobibmacro{author}{\egroup}{}{}

%keine Klammern um Jahr bei Article
\renewbibmacro*{issue+date}{%
	\setunit{\addcomma\space}%
	%
	\iffieldundef{issue}
	{\usebibmacro{date}}
	{\printfield{issue}%
		\setunit*{\addspace}%
		%
		\usebibmacro{date}}
	\newunit}

%Autor-Name Reihenfolge
\DeclareNameAlias{sortname}{last-first}

%Semikolon als Trenner zwischen Autoren und kein "und"
\newcommand*{\bibmultinamedelim}{\addsemicolon\space}
\newcommand*{\bibfinalnamedelim}{\addsemicolon\space}
\AtBeginBibliography{
	\let\multinamedelim\bibmultinamedelim
	\let\finalnamedelim\bibfinalnamedelim
}

\addbibresource{refs/references.bib}

%Array für Tabellen
\usepackage{array}

%Transparent
\usepackage{transparent}

%Abbildung und Tabelle in den Verzeichnissen
\usepackage[titles]{tocloft}
\renewcommand{\cfttabpresnum}{Tabelle }
\renewcommand{\cftfigpresnum}{Abbildung }
\renewcommand{\cftfigaftersnum}{:}
\renewcommand{\cfttabaftersnum}{:}
\settowidth{\cfttabnumwidth}{Tabelle 10 \quad}
\settowidth{\cftfignumwidth}{Abbildung 10 \quad}
%Punkte im Inhaltsverzeichnis bei Section
\renewcommand\cftsecdotsep{\cftdotsep}


%Zeilenabstand definiern
\usepackage{setspace}

%TabularX
\usepackage{tabularx}


%Serifenlose Matheschrift
\usepackage{sfmath}

%Subfigures
\usepackage{subfig}

%Trennung bestimmter Wörter


% Hyperref:
% This package makes all references within your document clickable. By default, these references will become boxed and colored. This is turned back to normal with the \hypersetup command below.
%Dieses Paket sollte möglichst zuletzt geladen werden!
\usepackage{hyperref}
\hypersetup{colorlinks=false,pdfborder=0 0 0}

% Cleveref:
% This package automatically detects the type of reference (equation, table, etc.) when the \cref{} command is used. It then adds a word in front of the reference, i.e. Fig. in front of a reference to a figure. With the \crefname{}{}{} command, these words may be changed.
%Cleverref nach Hyperref Laden!
\usepackage{cleveref}
\crefname{equation}{Gleichung}{Gleichung}
\crefname{figure}{Abbildung}{Abbildungen}	
\crefname{table}{Tabelle}{Tabellen}
\crefname{section}{Kapitel}{Kapitel}
\newcommand{\crefrangeconjunction}{ bis }
\newcommand{\creflastconjunction}{ und }
\newcommand{\crefpairconjunction}{ und }






