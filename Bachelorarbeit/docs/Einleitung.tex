\section{Einleitung}\label{sec:Einleitung}
%
\subsection{Abstract}\label{subsec:Abstract}

Durch fortlaufende Entwicklung im Bereich der elektro-chemischen Energiespeicher in Form von Lithium-Ionen-Batterien (LIB’s) gewinnen diese als Energieträger in der Mobilitätbranche stetig an Relevanz. Sowohl in Elektrofahrzeugen als auch in Konzepten und Neuentwicklungen in der Light-Aircraft-Branche finden LIB’s immer mehr Anwendungen. Da die Energiedichte dieser Batteriezellen aktuell noch einen Bruchteil der Energiedichte von konventionellen Treibstoffen beträgt und die Ladezeiten aufgrund geringer C-Raten ein Vielfaches der zum Tanken von Treibstoff benötigten Zeit betragen, haben Verluste und Leistung, die für die Kühlung/Aufheizung der Energiespeicher aufgewendet werden muss, zusammen mit dem benötigten Gewicht für Kühlkreisläufe bei der Reichweite und Effizienz der Luft- und Kraftfahrzeuge einen signifikant negativen Effekt. Daher ist die optimale thermische Anbindung der Batteriezellen wünschenswert. Durch optimierte Wärmeleitung und geringerem Ohm’schen Widerstand können die Systemeffizienz erhöht und die Ladedauer reduziert werden. \\
In dieser Arbeit soll ein Konzept für eine thermisch optimierte prismatische Zelle aus bereits existierenden Konzepten für andere Zelltypen erarbeitet und dann anhand einer thermischen Simulation validiert werden. \\

\subsection{Aufbau der Arbeit}\label{subsec:Aufbau}
 


\newpage
\section{Motivation}\label{sec:Motivation}

Der Transportsektor trägt in der EU mit einem Anteil von ca. 25\% signifikant zu der gesamten Treibhausgasemission (THG) bei. Um die Ziele der EU-Kommission zu erreichen, soll bis 2030 die Anzahl der Fahrzeuge mit konventionellem Antrieb halbiert werden. Bis 2050 soll auf Fahrzeuge mit Benzin- oder Diesel-Motoren komplett verzichtet werden. Eines der Hindernisse für eine Marktdurchdringung der Fahrzeuge mit elektrochemischen Energiespeicher in Form von Lithium-Ionen-Batterien (LIB's) ist die begrenzte Reichweite dieser Fahrzeugklasse\footcite[Vgl.][S.136-146]{Ajanovic2020}.\\
Die Energiedichte von aktuellen LIB's\footcite[Vgl.\label{cite:Hettesheimer}][S. 11]{Hettesheimer2017} liegt nach Tabelle \ref{tab:Energiedichten} weit unter der von konventionellen Treibstoffen\footcite[Vgl.][]{BeloitEDU2021}, ist jedoch verglichen mit älteren Batterietechnologien erheblich höher\footcite[Vgl.][]{Sollmann2018}. Da auch die Batterie-Aufladezeiten ein Vielfaches der Dauer einer Tankfüllung mit einem Flüssigtreibstoff beträgt, haben Verluste die durch Widerstand oder Abwärme entstehen, zusammen mit Leistung die für das Kühlen bzw. Aufheizen der Batterie aufgewendet werden muss, einen signifikant negativen Effekt auf die Reichweite und Effizienz der Fahrzeuge.\\

\begin{table}[H]
	\caption{Vergleich der Energiedichten von Energieträgern in Fahrzeugen}
	\label{tab:Energiedichten}
	\vspace{0.2cm}	
	\begin{tabularx}{\textwidth}{ |X|X|X|  }
		\toprule[1.5pt]
		\textbf{Typ} & \textbf{Wert} & \textbf{Einheit} \\
		\hline\hline
		Lithium-Ionen-Batterie: & 430 - 800 & Wh/l \\
		\hline
		Nickel-Cadmium-Batterie: & 130 & Wh/l \\
		\hline
		Benzin: & 9700 & Wh/l \\
		\hline
		Diesel: & 10700 & Wh/l \\
		\bottomrule[1.5pt]
	\end{tabularx}		
\end{table}

Besonders deutlich ist dieser Effekt in der Luftfahrtbranche. \\
Die früheren Konzepte des elektrischen Fliegens waren zwar erfolgreich darin, dass sie das Fliegen mit elektrischem Antrieb ermöglichten, scheiterten jedoch an der geringen Energiedichte verfügbarer Energiespeicher wie Nickel-Cadmium-Batterien und der damit verbundenen möglichen Reichweite\footcite[Vgl.][S. 4]{Hepperle2012}.\\
Durch die Entwicklung von Lithium-Ionen-Batterietechnologie haben die elektrochemischen Energiespeicher erstmals eine ausreichende Energiedichte um relevante Reichweiten von bis zu 250km zu ermöglichen.\footcite[Vgl.][]{Lilium}.\\
Da die Reichweite sich antiproportional zum Gewicht verhält resultiert ein geringeres Gewicht bei gleicher Batteriekapazität in erweiterter Reichweite\footcite[Vgl.][S. 705]{Traub2011}.\\
Daher kann die Effizienz, Ladezeit und Reichweite von elektrischen Fahrzeugen mithilfe innovativer Kühlkonzepte gesteigert werden. In dieser Arbeit sollen diese Konzepte erarbeitet und validiert werden.\\

















%
%%\cref{fig:Blume} zeigt ein Beispiel für eine Abbildung.
%\begin{figure}[H]
%	\flushleft
%	\includegraphics[scale=0.2]{figs/Blume.jpeg}
%	\caption{dies ist eine schöne Blume}
%	\label{fig:Blume} 
%\end{figure}
%\noindent
%\cref{eq:Formel} zeigt ein Beispiel für eine Formel.
%\begin{align}\label{eq:Formel}
%	I = \int_{t_0}^{t_0+n \cdot h} f(t) \hspace{0.1cm} dt
%\end{align}
%\noindent
%\cref{tab:Tabelle} zeig ein Beispiel für eine Tabelle
%\begin{table}[H]
%	\flushleft
%	\caption{Tabelle}
%	\label{tab:Tabelle} 
%	\vspace{0.2cm}
%	\begin{tabularx}{\textwidth}{p{3.5cm}XXXXXXX}
%		\toprule[1.5pt]
%		ABC: & 1 & 2 & 3 \\
%		\midrule[1.0pt]
%		DEF: & 4 & 5 & 6 \\
%		\bottomrule[1.5pt]	
%	\end{tabularx}
%\end{table}
%\noindent
%Zitiert wird mit "footcite". Das sieht dann so %\footcite[Vgl.][S. 456]{PeterPan2017} oder so\footcite[Vgl.][S. 123]{DEF2014} aus.
%%
%\subsection{Unterkapitel 1}\label{subsec:Unterkapitel1}
%%
%%
%\subsection{Unterkapitel 2}\label{subsec:Unterkapitel2}
%%
%\subsubsection{Unterkapitel 2.1}\label{subsec:Unterkapitel2.1}
%%
%
%


