\section{Innovative Batteriezellkonzepte}\label{sec:innovativeBattery}

In diesem Kapitel werden zuerst die Inhalte des Tesla Patents nach Tsuruta et al. auf eine prismatische Zelle mit querliegenden Elektrodenwickel erweitert. Dann wird Anhand von Wärmeübertragungsmechanismen die effektivste Variation der innovativen Batteriegestaltungsmöglichkeiten für die Simulation ausgewählt.\\

\subsection{Lage des Elektrodenwickels}\label{sub:lageElektrode}

In der prismatischen Zelle kann der Elektrodenwickel waagerecht oder senkrecht installiert werden. Da die Stromabnahme an der Oberseite des Batteriegehäuses stattfindet, ist es von Vorteil die Flachwickeleletrode oder die Stapelelektrode waagerecht, wie in Abbildung \ref{fig:LageElektrodenwickel} dargestellt ist, in dem Batteriegehäuse einzubringen um die Distanz von Elektroden-Tab zum Anschluss zu minimieren.\footnote{Vgl. Fußnote \ref{cite:woehrle}, Wöhrle 2013}\\

\begin{figure}[H]
	\begin{center}
		\begin{overpic}[width=12cm]{figs/JellyRollPositionSchematic.eps}
			\put(15,192){Anschluss}
			\put(230,168){Elektrodenwickel}
			\put(80,5){Elektroden-Tab}
			\put(-30,113){Gehäuse}
		
		\end{overpic}
	
		\caption{Orientierung des Elektrodenwickels in der prismatischen Zelle}
	
		\label{fig:LageElektrodenwickel}
	\end{center}
\end{figure}


\subsection{Batteriegeometrie}\label{sub:batterygeometry}

Die Batteriegeometrie wird nach DIN 91252 dimensioniert.\footcite[Vgl. \label{cite:din91252}][]{DIN91252}\\
Für die Maße des Batteriegehäuses werden die Maße für die \textbf{BEV-1}-Batteriezelle gewählt, da die Batteriezellen aus dem Tesla Patent nach Tsuruta et al. für batterieelektrische Fahrzeuge ausgelegt ist.\footnote{Vgl. Fußnote \ref{cite:TeslaPatent}, Tsuruta et al. 2020} \\
In Abbildung \ref{fig:dimensionsCase} ist das Gehäuse der Batteriezelle abgebildet. Die Maße sind in Tabelle \ref{tab:caseDimensions} aufgeführt.

\begin{figure}[H]
	\begin{center}
		\begin{overpic}[width=12cm]{figs/Measurements_PrismaticCell.eps}
			\put(13,185){F}
			\put(100,185){E}
			\put(300,185){G}
			\put(100,4){A}
			\put(319,90){C}
			\put(329,90){D}
			\put(67,112){$E/2$}
			\put(280,4){B}
			\put(40,23){5}
			\put(158,23){5}
			\put(9,28){5}
			\put(9,128){5}
			\put(180,28){5}
			\put(180,128){5}
		\end{overpic}
		
		\caption{Vgl. Maße des Batteriegehäuses nach DIN 91252}
		
		\label{fig:dimensionsCase}
	\end{center}
\end{figure}

%%insert figure here

\begin{table}[H]
	\caption{Maße des Batteriegehäuses aus Abbildung \ref{fig:dimensionsCase}}
	\label{tab:caseDimensions}
	\vspace{0.2cm}
	\begin{tabularx}{\textwidth}{ |X|X|X|  }
		\toprule[1.5pt]
		\textbf{Bezeichung} & \textbf{Maß in [mm]} & \textbf{Umschreibung}\\
		\hline\hline
		A & 173 & Zelllänge \\
		\hline
		B & 32 & Zellbreite\\
		\hline
		C & 115 &  Zellhöhe ohne Anschlüsse\\
		\hline
		D & $\leq$ 123 & Zellhöhe mit Anschlüssen\\
		\hline
		E & 133 (sym) & Entfernung zwischen Anschlüssen\\
		\hline
		F & $\leq$ 24 & Anschlusslänge\\
		\hline
		G & $\leq$ 18,4 & Anschlussbreite\\
		\bottomrule[1.5pt]
	\end{tabularx}
\end{table}

Das Gehäuse wird aus Aluminium oder Edelstahl hergestellt. Zusammen mit den Separatoren, Stromableitern und dem Elektrolyt stellt das Gehäuse die passivene Anteile der Batterie dar.\footnote{Vgl. Fußnote \ref{cite:woehrle}, Wöhrle 2013, S. 111}\\
Für diese Arbeit wird für das Gehäuse Aluminium verwendet, da es aufgrund seiner geringeren Dichte einen Gewichtsvorteil gegenüber Edelstahl darstellt.\footcite[Vgl.][]{Edelstahlrohrshop.2021}\\
Der Elektrodenwickel wird so dimensioniert, dass bei der konventionellen und innovativen Batteriezelle die äußere Separatorschicht das Batteriegehäuse an den Seiten und der Unterseite berührt um eine maximale Kontaktkühlfläche zu gewährleisten. Die genauen Maße des konventionellen Elektrodenwickels sind in Abbildung \ref{fig:dimensionsJellyRoll} mit Tabelle \ref{tab:JellyRollSizeDescription} dargestellt.\\

\begin{figure}[H]
	\begin{center}
		\begin{overpic}[width=\linewidth]{figs/JellyRollMeasurements.eps}
			\put(180,148){$A_{j}$}
			\put(15,56){$E_{j}$}
			\put(352,56){$E_{j}$}
			\put(28,96){$D_{j}$}
			\put(335,96){$D_{j}$}
			\put(-7,56){$C_{j}$}
			\put(420,148){$B_{j}$}
			
		\end{overpic}
		
		\caption{Maße des konventionellen Elektrodenwickels}
		
		\label{fig:dimensionsJellyRoll}
	\end{center}
\end{figure}

\begin{table}[H]
	\caption{Maße des Elektrodenwickels der konventionellen Zelle aus Abbildung \ref{fig:dimensionsJellyRoll}}
	\label{tab:JellyRollSizeDescription}
	\vspace{0.2cm}
	\begin{tabularx}{\textwidth}{ |X|X|X|  }
		\toprule[1.5pt]
		\textbf{Bezeichung} & \textbf{Maß in [mm]} & \textbf{Umschreibung}\\
		\hline\hline
		$A_{j}$ & 140 & Elektrodenwickellänge\\
		\hline
		$B_{j}$ & 22 & Elektrodenwickelbreite\\
		\hline
		$C_{j}$ & 100 &  Elektrodenwickelhöhe\\
		\hline
		$D_{j}$ & 8 & Stromableitertabbreite\\
		\hline
		$E_{j}$ & 8 & Stromableitertabhöhe\\
		\hline
		\bottomrule[1.5pt]
	\end{tabularx}
\end{table}

Um den Strom zu den Anschlüssen zu leiten wird eine Aluminiumkonstruktion benutzt. Zwar hat Kupfer gegenüber Aluminium eine bessere Wärme- und Stromleitfähigkeit, besitzt jedoch eine größere Dichte und ist teurer in der Anschaffung.\footcite[Vgl.][]{Industr..2021}\\



\subsection{Ausarbeitung der innovativen Konzepte auf Zellebene}\label{sub:ausarbeitungKonzept}

Im Folgenden wird zuerst das Entwicklungsvorgehen beschrieben, anschließend werden innovative Konzepte anhand des Vorgehens ausgearbeitet. Zuletzt wird ein Konzept anhand definierter Kriterien und der Anforderungen an das System ausgewählt und dann als Simulationsmodell aufgebaut.

\subsubsection{Aufstellen des Vorgehens nach VDI 2206}\label{subsub:vorgehennachVDI}

Um effizientes Vorgehen bei dem Entwurf der innovativen Batteriezellkonzepte zu ermöglichen, wird die VDI Richtlinie 2206 herangezogen. Die Richtlinie hat den Namen \textbf{Entwicklungsmethodik für mechatronische Systeme} und wurde 2004 vom VDI-Ausschuss A127 erarbeitet.\footcite[Vgl.\label{cite:vdi2206}][S. 3]{VDI2206.June2004}\\
Die Motivation der VDI 2206 ist das Entwickeln von produktionsreifen Systemen, jedoch handelt diese Arbeit von einer Konzeptentwicklung und -bewertung. Daher werden nur bestimmte Aspekte der Richtlinie in betracht gezogen.\\

\subsubsection*{Das V-Modell}

Dsa V-Modell beschreibt eine allgemeine Vorgehensweise beim Entwurf von Systemen, welches für jede individuelle Problemstellung entsprechend leicht angepasst werden muss.\\
Es ist gegliedert in 6 Unterpunkte, die zusammen ein iteratives Entwicklungsvorgehen beschreiben. Den Ausgangspunkt bilden die Systemanforderungen. Die Anforderungen sind auch der spätere Bewertungsmaßstab der Entwicklung.\\
Im Systementwurf findet die Festlegung eines domänenübergreifenden Lösungskonzeptes statt. Um die logischen und physikalischen Mechanismen des Systems zu beschreiben, wird die Gesamtfunktion des Systems in Teilfunktionen zerlegt.\\
Diese Teilfunktionen werden dann im domänenspezifischen Entwurf ausgelegt. Danach werden die Teilfunktionen in der Systemintegration wieder zu dem Gesamtsystem integriert. Während dem Entwurf und der Systemintegration muss anhand der Eigenschaftsabsicherung fortlaufend der Entwurfsfortschritt mit den Anforderungen abgeglichen werden.\\
Über die Modellbildung und -analyse ist das Endergebnis des durchlaufenen Makrozyklus das Endprodukt oder -system. Meist sind für das Endprodukt mehrere Durchläufe dieses Zyklus nötig.\footnote{Vgl. Fußnote \ref{cite:vdi2206}, VDI 2206 - Entwicklungsmethodik für mechatronische Systeme, S. 29-30}\\
In Abbildung \ref{fig:VModel} ist das V-Modell abgebildet. Die benötigten Schritte für den Entwurf der innovativen Konzepte und das angepasste V-Modell werden im Folgenden beschrieben.

\begin{figure}[H]
	\begin{center}
		\begin{overpic}[width=\linewidth]{figs/VModelAdjustedForBA.eps}
			\put(30,304){\textbf{Anforderungen}}
			\put(330, 304){\textbf{Innovatives Konzept}}
			\put(158,4){\textbf{Modellbildung und -analyse}}
			\put(162,185){\textbf{Eigenschaftsabsicherung}}
			\put(180,57){\textbf{Domänenspezifischer}}
			\put(215,45){\textbf{Entwurf}}
			\put(85,200){\rotatebox{282.23}{\textbf{Systementwurf}}}
			\put(342,130){\rotatebox{77.77}{\textbf{Systemintegration}}}
		\end{overpic}
		
		\caption{Vgl. V-Modell nach VDI 2206}
		
		\label{fig:VModel}
	\end{center}
\end{figure}

\newpage 
\underline{Anforderungen an das innovative Konzept}\\
\\

Im Vergleich zur konventionellen Rundzelle beschreiben Tsuruta et al. in ihrem Patent einen geringeren Wärmegradienten in der neuen Zelle und dem Elektrodenwickel, zusammen mit einer verbesserten Kühlung der Lithium-Ionen Batterie.\footnote{Vgl. Fußnote \ref{cite:TeslaPatent}, Tsuruta et al. 2020}\\
Die Anforderungen an das innovative Konzept in der prismatischen Zelle gestalten sich ähnlich. Konkret sollte das Konzept größere Wärmeleitung innerhalb der Batteriezelle und dem Elektrodenwickel ermöglichen. Hierdurch soll mit entsprechender Kühlung der Wärmegradient reduziert und somit die Alterungsmechanismen im Wickel vermindert werden.\\
Zudem soll durch die effizientere Wärmedissipation eine höhere Leistungsabnahme ermöglicht werden.\\
Zuletzt werden die ökonomischen Aspekte des Konzepts betrachtet. Die Herstellungskosten und Herstellungsprozesskomplexität der innovativen prismatischen Batteriezelle sollten gleich zur konventionellen Batteriezelle bleiben oder verringert werden, jedoch sind geringe Anstiege der beiden Herstellungsparameter auch vertretbar insofern das innovative Konzept diese negativen Effekte durch seine verbesserten thermischen Eigenschaften ausgleicht.\\

\underline{Entwurf der innovativen Konzepte}\\
\\

Das System ist in diesem Fall das innovative Batteriezellkonzept. Da das System in dem Batteriegehäuse ist, unterliegt es geometrischen Einschränkung die bei der Konzeptionierungsphase berücksichtigt werden müssen. Die Rahmenbedingungen werden wie folgt festgelegt: \\

\begin{itemize}
	\item Das Konzept soll so leicht wie möglich sein, um eine signifikante Gewichtszunahme des Fahrzeugs, aufgrund des negativen Effekts den diese auf die Reichweite hat, zu vermeiden.
	\item Das Konzept soll so kostengünstig wie möglich sein.
	\item Das Konzept soll den geringsten Komplexitätsgrad besitzen, mit dem es die Anforderungen noch erfüllt.
\end{itemize}

\underline{Domänenspezifischer Entwurf}\\
\\

Für den Entwurf müssen folgende Domänen bearbeitet werden: 

\begin{itemize}
	\item Materialspezifische Domäne
	\subitem Hier wird betrachtet, welche Materialien für die innovativen Konzepte in Frage kommen. Kriterien sind die Anschaffungskosten, Dichte, sowie die Wärmeleitfähigkeit und die Leitfähigkeit.
	\item Wärmeleittechnische Domäne
	\subitem Hier wird betrachtet, wie effizient das innovative Konzept die entstehende Wärme aus dem Elektrodenwickel zum Kühlmechanismus transportiert.
	\item Produktionstechnische Domäne
	\subitem Hier wird betrachtet, wie das innovative Konzept den Produktionsprozess von Lithium-Ionen Batterien beeinflusst.
\end{itemize}

Außerdem werden hier die entwickelten Lösungsansätze vertieft und detaillierter ausgeführt.\\

\underline{Systemintegration}\\
\\

Hier werden die domänenspezifische Lösungsansätze wieder zu einem Gesamtsystem zusammengefügt. Gleichzeitig wird durchgehend die Eigenschaftsabsicherung der Lösungsansätze durchgeführt um die Anforderungen des Gesamtsystems zu erfüllen.\\
Des weiteren wird hier ein innovatives Konzept anhand des Erfüllungsgrad der Anforderungen ausgewählt und weiterführend als Simulationsmodell aufgebaut. Dieses wird anschließend mit der Simulation der konventionellen Zelle verglichen.\\

\subsubsection{Kühlkonzepte auf Batteriepaketebene}\label{subsub:coolingforBatterypacks}

Lithium-Ionen Batterien werden in der Anwendung durch verschiedene Methoden gekühlt. Unter anderem kommen Phasenwechselmaterialien, Luft- und Flüssigkühlung und Wärmerohre zum Einsatz.\footcite[Vgl.][S. 1,2]{Mohammed.2018}\\
Im diesem Kapitel werden die beiden dominanten Kühlkonzepte für prismatische Batteriezellen auf Akkumulatorebene vorgestellt.\\
Nach Darcovich et al. gibt es das sogenannte ``ice-plate`` Kühlkonzept, weiterhin das \textbf{IP}-Konzept, bei dem die Kühlplatten zwischen die prismatischen Zellen eingefügt werden und so die Seite des Batteriegehäuses kühlen, und das ``cold-plate`` Kühlkonzept, weiterhin das \textbf{CP}-Konzept, bei dem die Platte auf der die Batteriezellen stehen gekühlt wird.\footcite[Vgl.\label{cite:darcovich2019}][S. 186-187]{Darcovich.2019}\\
Das IP-Konzept erreicht im Vergleich mit CP-Konzept einen geringeren Temperaturgradienten innerhalb der Zelle. Jedoch ist es in der Herstellung komplizierter und ermöglicht geringere Kühlmittelströme durch die Platte. Um das IP-Konzept zu verwirklichen, muss die Kühlmittelströmung durch die Platte optimiert werden. Jedoch ist das IP-Konzept im Allgemeinen wirksamer als das CP-Konzept.\footnote{Vgl. Fußnote \ref{cite:darcovich2019}, Darcovich et al. 2019, S. 185}\\
Da das CP-Konzept einfacher umzusetzen ist, werden die Batteriezellkonzepte vorerst für diese Kühlmethode ausgelegt. Die vorgestellten Zelländerungen können allerdings an das IP-Konzept angepasst werden. \\


\subsubsection{Innovative Batteriezellkonzepte}\label{subsub:innovativeBatteriezellkonzepte}

Da bei der CP-Kühlmethode die Unterseite der Batteriezelle gekühlt wird und die Höhe des Batteriezellgehäuses nach DIN 91252 größer als die Breite ist, findet nach Gleichung \ref{gl:HeatTransferEquation} weniger Wärmeübertragung aus der Batteriezelle statt\footnote{Vgl. Fußnote \ref{cite:TeslaPatent}, Tsuruta et al. 2020}. Die Unterseitenfläche des Gehäuses ist kleiner als die Seitenfläche, die Höhe größer als die Breite, wodurch die Übertragungsdistanz  $d$ im Nenner und die Fläche $A$ in Gleichung \ref{gl:HeatTransferEquationInnovativ} beim CP-Konzept größer, bzw. kleiner sind und die Wärmeleitung $\dot{Q}$ innerhalb der Zelle bei gleichbleibendem Wärmeübergangskoeffizienten $k$ und Temperaturgradient $(T_{2} - T_{1})$ verringert ist. Nach Böckh ist durch die kleinere Kontaktfläche auch die Wärmeübertragung von dem Batteriezellgehäuse zur Kühlplatte bei dem CP-Konzept verringert.\footnote{Vgl. Fußnote \ref{cite:bockh}, Böckh 2017}\\

\begin{equation}
	\dot{Q} = \frac{k * A * ( T_{2} - T_{1})}{d}
	\label{gl:HeatTransferEquationInnovativ}
\end{equation}

Da jedoch nach Darcovich et al. das CP-Konzept einfacher und kostengünstiger zu implementieren ist, ist es das Ziel der innovativen Konzepte diesen Nachteil gegenüber des IP-Konzeptes, mit weniger ökonomischem Aufwand als das IP-Konzept darstellt, auszugleichen.\\
Daher ist das der erste Schritt der Konzeptentwicklung das Sicherstellen von besserer Wärmeleitung innerhalb der Batteriezelle. Für die Alterung und Degradierung der Batteriezelle sind Temperaturgradienten innerhalb des Elektrodenwickels, zusammen mit Hot-Spots im Wickel, die primären Auslöser.\footnote{Vgl. Fußnote \ref{cite:Waldmann2015}, Waldmann et al. 2015}\\

\subsubsection*{Entwurf der innovativen Konzepte}

Im Folgenden werden anhand von Wärmeübertragungsmechanismen und den Anforderungen aus Kapitel \ref{subsub:vorgehennachVDI} die innovativen Konzepte ausgearbeitet.\\
Da die Batteriezellen über Kontaktkühlung gekühlt sind, kann Wärme über das Gehäuse dissipiert werden. Ein Problem ist, dass der Elektrodenwickel und die Stromanschlüsse vor Kurzschlüssen geschützt werden müssen. Daher ist es notwendig, dass jegliches Anlehnung an das Tesla Patent von Tsuruta et al. von dem Gehäuse durch eine isolierende Schicht getrennt ist.\\
Nimmt man als Beispiel eine isolierende Schicht von Parker\textsuperscript{\textregistered} mit einer Durchschlagfestigkeit von $E_{ds} = 200\;VAC/mil$ und einer Dicke von $d = 1,016\;mm$ kann man mit Gleichungen \ref{gl:UmrechnungVACMILtoSI} und \ref{gl:Durchschlagfestigkeit} die Durchschlagfestigkeit dieser Schicht zu $V_{b} \approx 8000 V$ berechnen.\footcite[Vgl.\label{cite:ParkerPads}][]{Parker.2021}

\begin{equation}
	1 \frac{VAC}{mil} = 3,94 * 10^{4} \frac{V}{m}
	\label{gl:UmrechnungVACMILtoSI}
\end{equation}

\begin{equation}
	V_{b} = d * E_{ds}
	\label{gl:Durchschlagfestigkeit}
\end{equation}

Da Lithium-Ionen Batteriezellen mit Spannungen deutlich unter $V = 8000 V$ arbeiten, kann daher mit einer solchen Schicht die Isolation zwischen leitenden Teilen und Gehäuse sichergestellt werden. Die isolierende Schicht hat weiterhin eine Wärmeleitfähigkeit von $6,5 \frac{W}{m * K}$, wodurch das Abführen der Wärme aus der Batteriezelle an das Gehäuse gewährleistet wird.\footnote{Vgl. Fußnote \ref{cite:ParkerPads}, Thermally Conductive Pads von Parker} Um die Wärme innerhalb des Gehäuses zu leiten bieten sich Aluminium oder Kupfer als Materialien an. Da Aluminium leichter und kostengünstiger ist stellt es die bessere Alternative dar.\\
Nach Tsuruta et al. wird die entstehende Wärme aus dem Elektrodenwickel durch die Elektroden an das Gehäuse geleitet. Im zylindrischen Design findet dies im Patent nur an einer der beiden Elektroden statt\footnote{Vgl. Fußnote \ref{cite:TeslaPatent}, Tsuruta et al. 2020}. In der prismatischen Zelle bieten sich nach Abbildung \ref{fig:LageElektrodenwickel} jedoch die Kathode und die Anode an um die Wärme aus dem Elektrodenwickel an das Gehäuse und damit zum Kühlmechanismus zu Übertragen. Aufgrund der relativen Symmetrie der Batteriezelle ist es möglich auf beiden Seiten des Elektrodenwickels die gleichen Bauteile anzubringen.\\

\underline{Wärmeleitung innerhalb der Zelle}\\
\\

Wie im Tesla Patent sollen auch im Konzept für die prismatische Zelle die Anode und Kathode im Elektrodenwickel verlängert werden um eine Verbindung zu einer $Cap$ die als Stromkollektor dient zu ermöglichen. Analog zum Patent gibt es mehrere Möglichkeiten dies zu ermöglichen. Wie in Abbildung \ref{fig:CapGrooved} können in die $Cap$ Nuten gefräst werden, in welchen die leitenden Teile der Elektroden wie in Abbildung \ref{fig:BottomConnection} \textbf{a)} durch Anpresskraft oder Ultraschallschweißen eingebracht werden.\\
Von der $Cap$ aus soll nun durch Wärmeleitung zur Unterseite des Batteriegehäuses die Wärme zum Kühlmechanismus abgeführt werden. Hierfür kommt ein Verbindungsstück aus Aluminium in Frage. Um die Herstellungskomplexität zu verringern, kann das gesamte Stromabnehmerteil aus einem Stück Metall hergestellt werden. Mit einer L-Form nach Abbildung \ref{fig:CurrentCollectorInno} kann dies realisiert werden.\\

\begin{figure}[H]
	\begin{center}
		\begin{overpic}[width=\linewidth]{figs/LShapeConnectorInnovative.eps}
			\put(138,346){Nuten}
			\put(280,38){Wärmeleitender Teil zur Unterseite}
		\end{overpic}
		
		\caption{L-Form des Stromkollektors}
		
		\label{fig:CurrentCollectorInno}
	\end{center}
\end{figure}

Für das IP-Konzept kann der wärmeleitende Teil an den Seiten der $Cap$ befestigt sein. Das Prinzip der Wärmeleitung aus der Zelle bleibt hierbei gleich. \\
Aluminium hat eine mit zunehmender Temperatur abnehmende Wärmeleitfähigkeit\footcite[Vgl.][]{SchweizerFN.2021}. Jedoch ist im relevanten Temperaturbereich von -20 $\degreeCelsius$ bis 60 $\degreeCelsius$ die Wärmeleitfähigkeit $\lambda_{al} \approx 220 \frac{W}{m \cdot K}$ wesentlich höher als die der isolierenden Zwischenschicht. Daher ergibt es Sinn, den Stromkollektor mit so viel Fläche wie möglich an der unteren Seite des Gehäuses mit der Isolierschicht in Verbindung zu bringen. Würde der Stromkollektor aus zwei Teilen bestehen oder an dem Knick der L-Form getrennt, würde eine Engstelle entstehen. In dieser Engstelle kann aufgrund der Isolierschicht weniger Wärme übertragen werden. Daher sind Variationen mit diesen Baueigenschaften nicht wünschenswert.\\
Der Stromkollektor kann im Prinzip frei dimensioniert werden, allerdings darf er nicht den Platz des Elektrodenwickels einschränken, da dies Kapazitätsverluste verursacht.\\

\underline{Variationen des vorgestellten Konzepts}\\
\\

Mit gleichbleibender Wirkweise können nur wenige Parameter variiert werden. \\
Um die Herstellungskomplexität eines massiven Aluminiumkollektors zu senken, kann die $Cap$ separat von der L-Form produziert werden. Dann kommt die isolierende Schicht zwischen Rückseite der $Cap$ und dem Gehäuse. Somit würde das Gehäuse selbst die Wärme aus der Batteriezelle leiten, anstatt eines separaten Bauteils im Inneren der Batterie. Die benötigte Anpresskraft kann durch die Isolierschicht eingebracht werden. Somit wird auch die Komplexität der Bauteile innerhalb des Batteriegehäuses reduziert und der Herstellungsprozess vereinfacht.\\
Eine weitere Variation wäre anstatt des Aluminiums innerhalb der Zelle Kupfer zu verwenden. Die Wärmeleitfähigkeit von Kupfer liegt für Handelsware zwischen 240 und 380 $\frac{W}{m \cdot K}$\footcite[Vgl.][]{Myers.2009}. Da Aluminium in der Anschaffung um den Faktor 10 günstiger ist als Kupfer, die Wärmeleitfähigkeit von Kupfer jedoch weniger als doppelt so groß ist, stellt Kupfer keine relevante Alternative dar\footcite[Vgl.][]{Doduco.2021}.\\




\newpage
------------------------------\\
xa\\%%insert this somewhere else

In den Batteriepacks mit prismatischen Lithium-Ionen-Batterien bietet sich die Boden-Kontaktkühlung an, da die Batteriezellen aufgrund ihrer Form platzsparend nebeneinandner geschachtelt werde können. Zum Einsatz kommen Kühlplatten, auf denen die Batteriezellen stehen. %% QUelle finden!
Dadurch wird mit Hilfe der Kontaktkühlung auch das Jelly-Roll gekühlt, jedoch bildet sich aufgrund der einseitigen Kühlung ein Wärmegradient in der Zelle und dem Elektrodenwickel aus.\footcite[Vgl.][S. 2107]{Inui.2007}

\newpage
\section{Simulationsvorbereitung}\label{sec:SimulationPREP}

in diesem Kapitel wird zuerst das Referenzmodell der prismatischen Lithium-Ionen-Batterie erarbeitet. Danach wird das in Kapitel \ref{sec:innovativeBattery} ausgewählte innovative Konzept als Simulationsmodell erstellt. Im Anschluss werden beide Batterietypen anhand eines Lastprofils in COMSOL Multiphysics\textsuperscript{\textregistered} simuliert.\\



Der Elektrodenwickel im Inneren des Batteriegehäuses wird so dimensioniert, dass die äußere Separatorschicht die Innenwände des Gehäuses berührt.

Hier fange ich an über das Referenzmodell zu schreiben


	